% Copyright 2004 by Till Tantau <tantau@users.sourceforge.net>.
%
% In principle, this file can be redistributed and/or modified under
% the terms of the GNU Public License, version 2.
%
% However, this file is supposed to be a template to be modified
% for your own needs. For this reason, if you use this file as a
% template and not specifically distribute it as part of a another
% package/program, I grant the extra permission to freely copy and
% modify this file as you see fit and even to delete this copyright
% notice.

\RequirePackage{snapshot}
\documentclass{beamer}

%% The graphicx package provides the includegraphics command.
\usepackage{graphicx}

%% The amsthm package provides extended theorem environments
%% \usepackage{amsthm}
\usepackage{newtxtext,newtxmath,amsmath,amstext,amssymb}
\usepackage{xspace}

\usepackage{multicol}
\usepackage{multirow}

\usepackage{appendixnumberbeamer}


%------------------------------------------------------------------------------------
%----------------------  No values in newcommands below this line -------------------
%------------------------------------------------------------------------------------
% %
\newcommand{\oned         }{\ensuremath{\mbox{1-dimensional}}\xspace}
\newcommand{\twod         }{\ensuremath{\mbox{2-dimensional}}\xspace}
\newcommand{\threed       }{\ensuremath{\mbox{3-dimensional}}\xspace}
% %


\newcommand{\pileup       }{\ensuremath{\mbox{pile-up}}\xspace}
\newcommand{\Cutbased     }{\ensuremath{\mbox{Cut-based}}\xspace}
\newcommand{\cutbased     }{\ensuremath{\mbox{cut-based}}\xspace}
\newcommand{\Mvabased     }{\gls{BDT}}
\newcommand{\mvabased     }{\gls{BDT}}
\newcommand{\SelPurity    }{Selection purity}
\newcommand{\selPurity    }{selection purity}
% %
\newcommand{\XZ           }[3]{\ensuremath{#1\,\pm #2~\mathrm{(stat)}\,\pm #3~\mathrm{(syst)}}\xspace}
\newcommand{\XZst         }[2]{\ensuremath{#1\,\pm #2~\mathrm{(stat)}}\xspace}
\newcommand{\XZtot        }[2]{\ensuremath{#1\,\pm #2}\xspace}
% %
\newcommand{\ttbar        }{\ensuremath{t\bar{t}}\xspace}
\newcommand{\ttbarto      }{\ensuremath{\ttbar\to}\xspace}
\newcommand{\ttbarjj      }{\ensuremath{\ttbar\to\mbox{all-jets}}\xspace}
\newcommand{\ttbarll      }{\ensuremath{\ttbar\to\mbox{dilepton}}\xspace}
\newcommand{\ttbarlj      }{\ensuremath{\ttbar\to\mbox{lepton+jets}}\xspace}
\newcommand{\ttbarj       }{\ensuremath{t\bar{t}+1\mbox{~jet}}\xspace}
\newcommand{\pp           }{\ensuremath{\ensuremath{pp}}\xspace}
\newcommand{\ppbar        }{\ensuremath{\ensuremath{p\bar{p}}}\xspace}
\newcommand{\ppttbar      }{\ensuremath{\pp\to\ttbar}\xspace}
\newcommand{\WWbb         }{\ensuremath{W W b b}\xspace}
\newcommand{\WWbbplusminus}{\ensuremath{W^{+} W^{-} b \bar{b}}\xspace}
% \newcommand{\ttWWbb       }{\ensuremath{tt \to \WWbb}\xspace}
\newcommand{\ppttbarWWbb  }{\ensuremath{\pp\to tt \to \WWbb}\xspace}
\newcommand{\ppWWbb       }{\ensuremath{\pp\to \WWbb}\xspace}
\newcommand{\ljets        }{\ensuremath{l\mbox{+jets}}\xspace}
\newcommand{\dileptonic   }{\ensuremath{\mbox{dileptonic}}\xspace}
\newcommand{\semileptonic }{\ensuremath{\mbox{semileptonic}}\xspace}
\newcommand{\dil          }{\ensuremath{\mbox{dilepton}}\xspace}
\newcommand{\Dil          }{\ensuremath{\mbox{Dilepton}}\xspace}
\newcommand{\allh         }{\ensuremath{\mbox{all-hadronic}}\xspace}
\newcommand{\ee           }{\ensuremath{\mbox{ee}}\xspace}
\newcommand{\emu          }{\ensuremath{\mbox{e$\mu$}}\xspace}
\newcommand{\mumu         }{\ensuremath{\mbox{$\mu\mu$}}\xspace}
\newcommand{\ellell       }{\ensuremath{\mbox{\ell\ell}}\xspace}
\newcommand{\Wj           }{$W$+jets\xspace}
\newcommand{\Zj           }{$Z$+jets\xspace}
\newcommand{\Gammaj       }{$\gamma$+jets\xspace}
\newcommand{\abseta       }{\ensuremath{|\eta|}\xspace}
\newcommand{\pt           }{\ensuremath{p_\mathrm{T}}\xspace}
\newcommand{\px           }{\ensuremath{p_\mathrm{x}}\xspace}
\newcommand{\py           }{\ensuremath{p_\mathrm{y}}\xspace}
\newcommand{\pz           }{\ensuremath{p_\mathrm{z}}\xspace}
\newcommand{\Gt           }{\ensuremath{\Gamma_{\mathrm{top}}}\xspace}
% %\newcommand{\GW           }{\ensuremath{\Gamma_{\mathrm{W}}}\xspace}
\newcommand{\mt           }{\ensuremath{m_{\mathrm{top}}}\xspace}
\newcommand{\mtseven      }{\ensuremath{m_{\mathrm{top}}^{\mathrm{7~\TeV}}}\xspace}
\newcommand{\mtdl         }{\ensuremath{m_{\mathrm{top}}^{\mathrm{dil}}}\xspace}
\newcommand{\dmtdl        }{\ensuremath{\Delta m_{\mathrm{top}}^{\mathrm{dil}}}\xspace}
\newcommand{\dmtlj        }{\ensuremath{\Delta m_{\mathrm{top}}^{\mathrm{l+jets}}}\xspace}
\newcommand{\mtlj         }{\ensuremath{m_{\mathrm{top}}^{\mathrm{l+jets}}}\xspace}
\newcommand{\mtcb         }{\ensuremath{m_{\mathrm{top}}^{\mathrm{comb}}}\xspace}
\newcommand{\mtall        }{\ensuremath{m_{\mathrm{top}}^{\mathrm{all}}}\xspace}
\newcommand{\mtin         }{\ensuremath{\mt^{\mathrm{in}}}\xspace}
\newcommand{\mtout        }{\ensuremath{\mt^{\mathrm{out}}}\xspace}
\newcommand{\mtpole       }{\ensuremath{\mt^{\mathrm{pole}}}\xspace}
\newcommand{\mtmsbar      }{\ensuremath{\mt^{\mathrm{\gls{MSbar}}}}\xspace}
\newcommand{\mtMC         }{\ensuremath{\mt^{\mathrm{\gls{MC}}}}\xspace}
\newcommand{\etdr         }{\ensuremath{E_{T}^{\Delta R}}\xspace}
\newcommand{\mlb          }{\ensuremath{m_{\ell b}}\xspace}
\newcommand{\mlbplus      }{\ensuremath{m_{\ell^+ b}}\xspace}
\newcommand{\mlbminus     }{\ensuremath{m_{\ell^- b}}\xspace}
\newcommand{\mlbri        }{\ensuremath{m_{\ell b}^{\mathrm{reco,i}}}\xspace}
\newcommand{\mlbr         }{\ensuremath{m_{\ell b}^{\mathrm{reco}}}\xspace}
\newcommand{\mlbt         }{\ensuremath{m_{\ell b}^{\mathrm{truth}}}\xspace}
\newcommand{\ptlb         }{\ensuremath{p_{T,\ell b}}\xspace}
\newcommand{\ptlbt        }{\ensuremath{p_{T,\ell b}^{\mathrm{truth}}}\xspace}
\newcommand{\ptll         }{\ensuremath{p_{T,\ell \ell}}\xspace}
\newcommand{\ptbb         }{\ensuremath{p_{T,b b}}\xspace}
\newcommand{\mttwo        }{\ensuremath{m_{\mathrm{T2}}}\xspace}
\newcommand{\mtr          }{\ensuremath{\mt^{\mathrm{reco}}}\xspace}
\newcommand{\mtri         }{\ensuremath{\mt^{\mathrm{reco,i}}}\xspace}
\newcommand{\mW           }{\ensuremath{m_{\mathrm{W}}}\xspace}
\newcommand{\mll          }{\ensuremath{m_{\ell\ell}}\xspace}
\newcommand{\mH           }{\ensuremath{m_{\mathrm{H}}}\xspace}
\newcommand{\mWr          }{\ensuremath{m_{\mathrm{W}}^{\mathrm{reco}}}\xspace}
\newcommand{\mWri         }{\ensuremath{m_{\mathrm{W}}^{\mathrm{reco,i}}}\xspace}
\newcommand{\rbqr         }{\ensuremath{R_{\mathrm{bq}}^{\mathrm{reco}}}\xspace}
\newcommand{\rbqri        }{\ensuremath{R_{\mathrm{bq}}^{\mathrm{reco,i}}}\xspace}
\newcommand{\dR           }{\ensuremath{\Delta R}\xspace}
\newcommand{\mWt          }{\ensuremath{m_{\mathrm{T}}^{W}}\xspace}
\newcommand{\Ht           }{\ensuremath{H_{\mathrm{T}}}\xspace}
\newcommand{\hdamp        }{\ensuremath{h_{\mathrm{damp}}}\xspace}
% %
% %units
\newcommand{\eV             }{\ensuremath{\mathrm{eV}\xspace}}
\newcommand{\MeV            }{\ensuremath{\mathrm{MeV}\xspace}}
\newcommand{\GeV            }{\ensuremath{\mathrm{GeV}\xspace}}
\newcommand{\TeV            }{\ensuremath{\mathrm{TeV}\xspace}}
\newcommand{\PeV            }{\ensuremath{\mathrm{PeV}\xspace}}
\newcommand{\fb             }{\ensuremath{\mathrm{fb}\xspace}}
\newcommand{\pb             }{\ensuremath{\mathrm{pb}\xspace}}
\newcommand{\nb             }{\ensuremath{\mathrm{nb}\xspace}}
\newcommand{\mb             }{\ensuremath{\mathrm{mb}\xspace}}
\newcommand{\invfb          }{\ensuremath{\mathrm{fb}^{-1}\xspace}}
\newcommand{\cm             }{\ensuremath{\mathrm{cm}\xspace}}
\newcommand{\mm             }{\ensuremath{\mathrm{mm}\xspace}}
\newcommand{\mum            }{\ensuremath{\mu\mathrm{m}\xspace}}
\newcommand{\neqcm          }{\ensuremath{\mathrm{n}_\mathrm{eq}~\cm^{-2}\xspace}}
\newcommand{\instlumi       }{\ensuremath{\cm^{-2}\mathrm{s}^{-1}\xspace}}

% %
% %-kinematics
\newcommand{\cme            }{center-of-mass energy\xspace}
\newcommand{\cmes           }{center-of-mass energies\xspace}
\newcommand{\met            }{\ensuremath{E_\mathrm{T}^\mathrm{miss}}\xspace}
\newcommand{\mettruth       }{\ensuremath{E_\mathrm{T, truth}^\mathrm{miss}}\xspace}
\newcommand{\metx           }{\ensuremath{E_\mathrm{x}^\mathrm{miss}}\xspace}
\newcommand{\mety           }{\ensuremath{E_\mathrm{y}^\mathrm{miss}}\xspace}
\newcommand{\et             }{\ensuremath{E_\mathrm{T}}\xspace}

\newcommand{\etaclus        }{\ensuremath{\eta_\mathrm{cluster}}\xspace}
\newcommand{\absetaclus     }{\ensuremath{\vert\etaclus\vert}\xspace}
% %
\newcommand{\nvtx         }{\ensuremath{n_\mathrm{vtx}}\xspace}
\newcommand{\meanmu       }{\ensuremath{\langle\mu\rangle}\xspace}
% %
% %
\newcommand{\chiq         }{\ensuremath{\chi^2}\xspace}
\newcommand{\chidof       }{\ensuremath{\chi^2/\mathrm{\glsname{ndf}}}\xspace}
\newcommand{\lumi         }{\ensuremath{\mathcal L}\xspace}
\newcommand{\resp         }{\ensuremath{\mathcal R}\xspace}
\newcommand{\order        }{\ensuremath{\mathcal O}\xspace}
\newcommand{\intlumi      }{\ensuremath{\int\!\!{\mathcal L}\mathrm{d}t}\xspace}
\newcommand{\sqrts        }{\ensuremath{{\sqrt{\mathrm{s}}}}\xspace}
\newcommand{\mur          }{\ensuremath{\mu_\mathrm{R}}\xspace}
\newcommand{\muf          }{\ensuremath{\mu_\mathrm{F}}\xspace}
\newcommand{\alphas       }{\ensuremath{\alpha_{s}}\xspace}
\newcommand{\kfac         }{\ensuremath{K}-factor\xspace}
\newcommand{\LambdaQCD    }{\ensuremath{\Lambda_\mathrm{\gls{QCD}}}\xspace}
\newcommand{\N            }[1]{\ensuremath{N_{\mathrm{#1}}}\xspace}
\newcommand{\Egam         }{\ensuremath{\it{egamma}}\xspace}
\newcommand{\Muon         }{\ensuremath{\it{muon}}\xspace}
\newcommand{\Njet         }[1]{\ensuremath{N^{\mathrm{#1}}_{\mathrm{jets}}}\xspace}
\newcommand{\fake         }{\gls{NP}/fake lepton\xspace}
% %
% % Slang
\newcommand{\xsec         }{cross-section\xspace}
\newcommand{\Xsec         }{Cross-section\xspace}
\newcommand{\ctag         }{\ensuremath{c\mbox{-tagging}}\xspace}
\newcommand{\ttag         }{\ensuremath{t\mbox{-tagging}}\xspace}
\newcommand{\btag         }{\ensuremath{b\mbox{-tagging}}\xspace}
\newcommand{\btagged      }{\ensuremath{b\mbox{-tagged}}\xspace}
\newcommand{\bflavoured   }{\ensuremath{b\mbox{-flavoured}}\xspace}
\newcommand{\cflavoured   }{\ensuremath{c\mbox{-flavoured}}\xspace}
\newcommand{\tjet         }{\ensuremath{\mathit{t}\mbox{-jet}}\xspace}
\newcommand{\bjet         }{\ensuremath{\mathit{b}\mbox{-jet}}\xspace}
\newcommand{\cjet         }{\ensuremath{\mathit{c}\mbox{-jet}}\xspace}
\newcommand{\ljet         }{light jet\xspace}
\newcommand{\Ljet         }{Light jet\xspace}
\newcommand{\kt           }{\ensuremath{k_\mathrm{t}}\xspace}
\newcommand{\antikt       }{anti-\kt}
\newcommand{\bhadron      }{\ensuremath{B\mbox{-hadron}}\xspace}

%particles
\newcommand{\uquark       }{up~quark\xspace}
\newcommand{\dquark       }{down~quark\xspace}
\newcommand{\cquark       }{charm~quark\xspace}
\newcommand{\squark       }{strange~quark\xspace}
\newcommand{\bquark       }{bottom~quark\xspace}
\newcommand{\tquark       }{top~quark\xspace}
\newcommand{\Uquark       }{Up~quark\xspace}
\newcommand{\Dquark       }{Down~quark\xspace}
\newcommand{\Cquark       }{Charm~quark\xspace}
\newcommand{\Squark       }{Strange~quark\xspace}
\newcommand{\Bquark       }{Bottom~quark\xspace}
\newcommand{\Tquark       }{Top~quark\xspace}
\newcommand{\Zbos         }{\ensuremath{Z}\xspace}
\newcommand{\Zboson       }{\Zbos~boson\xspace}
\newcommand{\Wbos         }{\ensuremath{W^{\pm}}\xspace}
\newcommand{\Wboson       }{\Wbos~boson\xspace}
\newcommand{\Wplusbos     }{\ensuremath{W^{+}}\xspace}
\newcommand{\Wplusboson   }{\Wplusbos~boson\xspace}
\newcommand{\Hboson       }{\ensuremath{\mbox{Higgs~boson}}\xspace}
\newcommand{\Higgs        }{Higgs\xspace}
\newcommand{\Hbos         }{\ensuremath{H}\xspace}
\newcommand{\Hportal      }{\ensuremath{\mbox{Higgs~portal}}\xspace}
\newcommand{\Hsector      }{\ensuremath{\mbox{Higgs~sector}}\xspace}
\newcommand{\Kmeson       }{\ensuremath{K\mbox{~meson}}\xspace}
\newcommand{\Upsilonmeson }{\ensuremath{\Upsilon\mbox{~meson}}\xspace}
\newcommand{\Jpsimeson    }{\ensuremath{J/\Psi\mbox{~meson}}\xspace}

%sparticles
\newcommand\tsquark{top~squark\xspace}
\newcommand\Tsquark{Top~squark\xspace}


% %tdanalysis part F.Balli 02/04/2013
% %

% %templates part F.Balli 02/04/2013
\newcommand{\fbkg         }{\ensuremath{f_{\mathrm{bkg}}}\xspace}
\newcommand{\Ptop         }{\ensuremath{P_{\mathrm{top}}}\xspace}
\newcommand{\Like         }{\ensuremath{\mathcal{L}_\mathrm{shape}}\xspace}
\newcommand{\Likedil      }{\ensuremath{\mathcal{L}_\mathrm{shape}^\mathrm{dilep}}\xspace}
\newcommand{\Likeljets    }{\ensuremath{\mathcal{L}_\mathrm{shape}^\mathrm{l+jets}}\xspace}
\newcommand{\Ptopsig      }{\ensuremath{P_{\mathrm{top}}^{\mathrm{sig}}}\xspace}
\newcommand{\Ptopbkg      }{\ensuremath{P_{\mathrm{top}}^{\mathrm{bkg}}}\xspace}
\newcommand{\PW           }{\ensuremath{P_{\mathrm{W}}}\xspace}
\newcommand{\PWsig        }{\ensuremath{P_{\mathrm{W}}^{\mathrm{sig}}}\xspace}
\newcommand{\PWbkg        }{\ensuremath{P_{\mathrm{W}}^{\mathrm{bkg}}}\xspace}
\newcommand{\Prbq         }{\ensuremath{P_{\mathrm{\mathcal{R}_{bq}}}}\xspace}
\newcommand{\Prbqsig      }{\ensuremath{P_{\mathrm{\mathcal{R}_{bq}}}^{\mathrm{sig}}}\xspace}
\newcommand{\Prbqbkg      }{\ensuremath{P_{\mathrm{\mathcal{R}_{bq}}}^{\mathrm{bkg}}}\xspace}

% % Programs
\newcommand{\Acermc       }{{\sc AcerMC}\xspace}
\newcommand{\Alpgen       }{{\sc Alpgen}\xspace}
\newcommand{\AlpgenHerwig }{{\sc Alpgen+Herwig}\xspace}
\newcommand{\Herwig       }{{\sc Herwig}\xspace}
\newcommand{\Herwigpp     }{{\sc Herwig++}\xspace}
\newcommand{\Jimmy        }{{\sc Jimmy}\xspace}
\newcommand{\Mcatnlo      }{{\sc MC@NLO}\xspace}
\newcommand{\McatnloHerwig}{{\sc MC@NLO+Herwig}\xspace}
\newcommand{\Pythia       }{{\sc Pythia}\xspace}
\newcommand{\PowhegPythia }{{\sc Powheg+Pythia}\xspace}
\newcommand{\PowhegHerwig }{{\sc Powheg+Herwig}\xspace}
\newcommand{\Pythiasix    }{{\sc Pythia6}\xspace}
\newcommand{\Pythiaeight  }{{\sc Pythia8}\xspace}
\newcommand{\Powheg       }{{\sc Powheg}\xspace}
\newcommand{\Sherpa       }{{\sc Sherpa}\xspace}
\newcommand{\Gosam        }{{\sc GoSam}\xspace}
\newcommand{\Madgraph     }{{\sc Madgraph}\xspace}
\newcommand{\Fastjet      }{{\sc FastJet}\xspace}
\newcommand{\Geantfour    }{{\sc Geant4}\xspace}
\newcommand{\Atlfast      }{{\sc AtlFast2}\xspace}
\newcommand{\Roounfold    }{{\sc RooUnfold}\xspace}
\newcommand{\TRUEE        }{\gls{TRUEE}\xspace}
\newcommand{\TMVA         }{{\sc TMVA}\xspace}
\newcommand{\Root         }{{\sc ROOT}\xspace}
\newcommand{\TRC          }{{\sc TopRootCore}\xspace}
\newcommand{\Cpp          }{{\sc C++}\xspace}

% % unfolding
\newcommand{\genl         }{generator\xspace}
\newcommand{\genlevel     }{\genl\ level\xspace}
\newcommand{\Genlevel     }{Generator level\xspace}
\newcommand{\stabl        }{stable particle\xspace}
\newcommand{\stablevel    }{\stabl\ level\xspace}
\newcommand{\Stablevel    }{Stable particle level\xspace}
\newcommand{\recol        }{reconstruction\xspace}
\newcommand{\recolevel    }{\recol\ level\xspace}
\newcommand{\Recolevel    }{Reconstruction level\xspace}
\newcommand{\truel        }{truth\xspace}
\newcommand{\truelevel    }{\truel\ level\xspace}
\newcommand{\Truelevel    }{Truth level\xspace}


%slang
\newcommand{\RunOne      }{Run-I\xspace}
\newcommand{\RunTwo      }{Run-II\xspace}
\newcommand{\RunThree    }{Run-III\xspace}
\newcommand{\blayer      }{B-Layer\xspace}
\newcommand{\layerzero   }{Layer 0\xspace}
\newcommand{\layerone    }{Layer 1\xspace}
\newcommand{\layertwo    }{Layer 2\xspace}


%lateX objects
\newcommand\fig{Figure\xspace}
\newcommand\sect{Section\xspace}
\newcommand\chap{Chapter\xspace}
\newcommand\tab{Table\xspace}
\newcommand\reference{Reference\xspace}
\newcommand\app{Appendix\xspace}
% \newcommand\eq{Equation\xspace}
\newcommand\Fig{Figure\xspace}
\newcommand\Sect{Section\xspace}
\newcommand\Chap{Chapter\xspace}
\newcommand\Tab{Table\xspace}
\newcommand\Reference{Reference\xspace}
\newcommand\App{Appendix\xspace}
% \newcommand\Eq{Equation\xspace}

%misc
\newcommand\HFoundation{Alexander von Humboldt-Foundation\xspace}
\newcommand\FLynenFellowship{Feodor Lynen-Fellowship\xspace}

%todo
\newcommand{\todo}[1]{
  \hspace{0pt}
  \marginpar{
    \color{red}\textbf{TODO}
    % \raggedright\sffamily\footnotesize #1
  }
  \footnote{\color{red} \bf{TODO: #1}}
  \externallink{#1}
  \color{black}
}

\newcommand{\figurepath}{../analysis/fig/}
\newcommand{\texpath}{../analysis/tex/}
\newcommand{\tripleFigDistance}{\vspace{-1.2em}}


% There are many different themes available for Beamer. A comprehensive
% list with examples is given here:
% http://deic.uab.es/~iblanes/beamer_gallery/index_by_theme.html
% You can uncomment the themes below if you would like to use a different
% one:
%\usetheme{AnnArbor}
%\usetheme{Antibes}
%\usetheme{Bergen}
%\usetheme{Berkeley}
%\usetheme{Berlin}
%\usetheme{Boadilla}
%\usetheme{boxes}
%\usetheme{CambridgeUS}
%\usetheme{Copenhagen}
%\usetheme{Darmstadt}
%\usetheme{default}
%\usetheme{Frankfurt}
%\usetheme{Goettingen}
%\usetheme{Hannover}
%\usetheme{Ilmenau}
%\usetheme{JuanLesPins}
%\usetheme{Luebeck}
\usetheme{Madrid}
%\usetheme{Malmoe}
%\usetheme{Marburg}
%\usetheme{Montpellier}
%\usetheme{PaloAlto}
%\usetheme{Pittsburgh}
%\usetheme{Rochester}
%\usetheme{Singapore}
%\usetheme{Szeged}
%\usetheme{Warsaw}

\title{Comparison of old and new detector model}

% A subtitle is optional and this may be deleted
\subtitle{The CLIC\_ILD vs the CLIC\_o3\_v14 model}

\author{Andreas A. Maier\inst{1}}
% - Give the names in the same order as the appear in the paper.
% - Use the \inst{?} command only if the authors have different
%   affiliation.

\institute[CERN] % (optional, but mostly needed)
{
  \inst{1}%
  CERN
  % \and
  % \inst{2}%
  % Department of Theoretical Philosophy\\
  % University of Elsewhere
}
% - Use the \inst command only if there are several affiliations.
% - Keep it simple, no one is interested in your street address.

\date{\today}
% - Either use conference name or its abbreviation.
% - Not really informative to the audience, more for people (including
%   yourself) who are reading the slides online

\subject{Particle Physics}
% This is only inserted into the PDF information catalog. Can be left
% out.

% If you have a file called "university-logo-filename.xxx", where xxx
% is a graphic format that can be processed by latex or pdflatex,
% resp., then you can add a logo as follows:

% \pgfdeclareimage[height=0.5cm]{university-logo}{university-logo-filename}
% \logo{\pgfuseimage{university-logo}}

% Delete this, if you do not want the table of contents to pop up at
% the beginning of each subsection:
\AtBeginSubsection[]
{
  \begin{frame}<beamer>{Outline}
    \begin{multicols}{2}
      \tableofcontents[currentsection,currentsubsection]
    \end{multicols}
  \end{frame}
}

% Let's get started
\begin{document}

% Title page
\begin{frame}
  \titlepage
\end{frame}


















% TOC
\begin{frame}{Outline}
  \begin{multicols}{2}
    \tableofcontents
  \end{multicols}
  % You might wish to add the option [pausesections]
\end{frame}









\section{Analysis details}

\begin{frame}{Datasets}
\input{\texpath/comp_samples.tex}
\vspace{-0.5cm}
Both datasets are based on the same MC events and differ solely in the detector model used for the reconstruction.

The events in the datasets are scaled with the luminosity (1.5 \invab) and their \xsec.
%
Additionally, they are divided by the total number of entries in the dataset.

\end{frame}


\begin{frame}{Physics objects}
The physics objects used in this analysis are:
\begin{itemize}
  %
  \item Leptons
  \begin{itemize}
    \item Isolated leptons
  \end{itemize}
  %
  \item Jets
  \begin{itemize}
    \item use Pandora PFOs without isolated leptons
    \item reconstructed using the Valencia alrogithm\footnote{algorithm string in marlin: `ValenciaPlugin 0.8 1 0.7'} with
    \begin{itemize}
      \item jet size $R=0.8$
      \item clustering order $\beta=1.0$
      \item jet shrinking for forward jets $\gamma=0.7$
    \end{itemize}
  \end{itemize}
  %
  \item Missing energy
  \begin{itemize}
    \item Initial state minus sum of all particle flow objects in `LooseSelectedPandoraPFANewPFOs`
    \item \pt\ and $\phi$ correlated with neutrino, $\theta$ and energy not. Does it come from the forward region? Better way to define it?
  \end{itemize}
  %
\end{itemize}
\end{frame}










\section{Event selection}

\begin{frame}{Event selection}
A basic event selection cuts has been applied to select the relevant physics objects:
\begin{itemize}
\item Exactly one charged lepton
\item Lepton $\pt > 10~\GeV$
\end{itemize}
\end{frame}









\section{Event yields}

\subsection{Single selection cuts}

\begin{frame}{Old $W\rightarrow\qqln$}
\input{\texpath/comp_cut_old.tex}
\end{frame}

\begin{frame}{New $W\rightarrow\qqln$}
\input{\texpath/comp_cut_new.tex}
\end{frame}









\section{Comparison figures}


\subsection{Events}

\begin{frame}{Beam energy}
\begin{figure}
\includegraphics[width=0.5\textwidth]{{\figurepath/comp_raw_beam_e}.pdf}
\includegraphics[width=0.5\textwidth]{{\figurepath/comp_raw_beam_m}.pdf}\\ \tripleFigDistance
\includegraphics[width=0.5\textwidth]{{\figurepath/comp_pre_beam_e}.pdf}
\includegraphics[width=0.5\textwidth]{{\figurepath/comp_pre_beam_m}.pdf}
\caption{
  \eplus\ \eminus\ beam energy and mass
}
\end{figure}
\end{frame}

% \begin{frame}{MC truth \Wboson\ masses}
% \begin{figure}
% \includegraphics[width=0.5\textwidth]{{\figurepath/comp_raw_mc_qq_m}.pdf}
% \includegraphics[width=0.5\textwidth]{{\figurepath/comp_raw_mc_ln_m}.pdf}\\ \tripleFigDistance
% \includegraphics[width=0.5\textwidth]{{\figurepath/comp_pre_mc_qq_m}.pdf}
% \includegraphics[width=0.5\textwidth]{{\figurepath/comp_pre_mc_ln_m}.pdf}
% \caption{
%   Masses of the two W objects at truth level.
% }
% \end{figure}
% \end{frame}



\subsection{Missing energy}

\begin{frame}{Missing energy kinematics}
\begin{figure}
\includegraphics[width=0.5\textwidth]{{\figurepath/comp_raw_miss_e}.pdf}
\includegraphics[width=0.5\textwidth]{{\figurepath/comp_raw_miss_pt}.pdf}\\ \tripleFigDistance
\includegraphics[width=0.5\textwidth]{{\figurepath/comp_pre_miss_e}.pdf}
\includegraphics[width=0.5\textwidth]{{\figurepath/comp_pre_miss_pt}.pdf}
\caption{
  Missing energy and \pt
}
\end{figure}
\end{frame}

\begin{frame}{Missing energy kinematics}
\begin{figure}
\includegraphics[width=0.5\textwidth]{{\figurepath/comp_raw_miss_theta}.pdf}
\includegraphics[width=0.5\textwidth]{{\figurepath/comp_raw_miss_phi}.pdf}\\ \tripleFigDistance
\includegraphics[width=0.5\textwidth]{{\figurepath/comp_pre_miss_theta}.pdf}
\includegraphics[width=0.5\textwidth]{{\figurepath/comp_pre_miss_phi}.pdf}
\caption{
  Missing energy $\phi$ and $\theta$
}
\end{figure}
\end{frame}




\subsection{Leptons}

\begin{frame}{All leptons}
\begin{figure}
\includegraphics[width=0.5\textwidth]{{\figurepath/comp_raw_lep_n}.pdf}
\includegraphics[width=0.5\textwidth]{{\figurepath/comp_raw_lep_type}.pdf}\\ \tripleFigDistance
\includegraphics[width=0.5\textwidth]{{\figurepath/comp_pre_lep_n}.pdf}
\includegraphics[width=0.5\textwidth]{{\figurepath/comp_pre_lep_type}.pdf}
\caption{
  Total lepton number and type
}
\end{figure}
\end{frame}

\begin{frame}{Total energy}
\begin{figure}
\includegraphics[width=0.5\textwidth]{{\figurepath/comp_raw_lep_etot}.pdf}\\ \tripleFigDistance
\includegraphics[width=0.5\textwidth]{{\figurepath/comp_pre_lep_etot}.pdf}
\caption{
  Total lepton energy
}
\end{figure}
\end{frame}

\begin{frame}{Lepton kinematics}
\begin{figure}
\includegraphics[width=0.5\textwidth]{{\figurepath/comp_raw_lep_e}.pdf}
\includegraphics[width=0.5\textwidth]{{\figurepath/comp_raw_lep_pt}.pdf}\\ \tripleFigDistance
\includegraphics[width=0.5\textwidth]{{\figurepath/comp_pre_lep_e}.pdf}
\includegraphics[width=0.5\textwidth]{{\figurepath/comp_pre_lep_pt}.pdf}
\caption{
  Lepton energy and \pt
}
\end{figure}
\end{frame}

\begin{frame}{Lepton kinematics}
\begin{figure}
\includegraphics[width=0.5\textwidth]{{\figurepath/comp_raw_lep_pt_0}.pdf}
\includegraphics[width=0.5\textwidth]{{\figurepath/comp_raw_lep_pt_1}.pdf}\\ \tripleFigDistance
\includegraphics[width=0.5\textwidth]{{\figurepath/comp_pre_lep_pt_0}.pdf}
\includegraphics[width=0.5\textwidth]{{\figurepath/comp_pre_lep_pt_1}.pdf}
\caption{
  Leading and sub-leading lepton \pt
}
\end{figure}
\end{frame}

\begin{frame}{Lepton kinematics}
\begin{figure}
\includegraphics[width=0.5\textwidth]{{\figurepath/comp_raw_lep_theta}.pdf}
\includegraphics[width=0.5\textwidth]{{\figurepath/comp_raw_lep_phi}.pdf}\\ \tripleFigDistance
\includegraphics[width=0.5\textwidth]{{\figurepath/comp_pre_lep_theta}.pdf}
\includegraphics[width=0.5\textwidth]{{\figurepath/comp_pre_lep_phi}.pdf}
\caption{
  Lepton $\phi$ and $\theta$
}
\end{figure}
\end{frame}


\begin{frame}{Electron kinematics}
\begin{figure}
\includegraphics[width=0.5\textwidth]{{\figurepath/comp_raw_el_e}.pdf}
\includegraphics[width=0.5\textwidth]{{\figurepath/comp_raw_el_pt}.pdf}\\ \tripleFigDistance
\includegraphics[width=0.5\textwidth]{{\figurepath/comp_pre_el_e}.pdf}
\includegraphics[width=0.5\textwidth]{{\figurepath/comp_pre_el_pt}.pdf}
\caption{
  Electron energy and \pt
}
\end{figure}
\end{frame}

\begin{frame}{Electron kinematics}
\begin{figure}
\includegraphics[width=0.5\textwidth]{{\figurepath/comp_raw_el_theta}.pdf}
\includegraphics[width=0.5\textwidth]{{\figurepath/comp_raw_el_phi}.pdf}\\ \tripleFigDistance
\includegraphics[width=0.5\textwidth]{{\figurepath/comp_pre_el_theta}.pdf}
\includegraphics[width=0.5\textwidth]{{\figurepath/comp_pre_el_phi}.pdf}
\caption{
  Electron $\phi$ and $\theta$
}
\end{figure}
\end{frame}


\begin{frame}{Muon kinematics}
\begin{figure}
\includegraphics[width=0.5\textwidth]{{\figurepath/comp_raw_mu_e}.pdf}
\includegraphics[width=0.5\textwidth]{{\figurepath/comp_raw_mu_pt}.pdf}\\ \tripleFigDistance
\includegraphics[width=0.5\textwidth]{{\figurepath/comp_pre_mu_e}.pdf}
\includegraphics[width=0.5\textwidth]{{\figurepath/comp_pre_mu_pt}.pdf}
\caption{
  Muon energy and \pt
}
\end{figure}
\end{frame}

\begin{frame}{Muon kinematics}
\begin{figure}
\includegraphics[width=0.5\textwidth]{{\figurepath/comp_raw_mu_theta}.pdf}
\includegraphics[width=0.5\textwidth]{{\figurepath/comp_raw_mu_phi}.pdf}\\ \tripleFigDistance
\includegraphics[width=0.5\textwidth]{{\figurepath/comp_pre_mu_theta}.pdf}
\includegraphics[width=0.5\textwidth]{{\figurepath/comp_pre_mu_phi}.pdf}
\caption{
  Muon $\phi$ and $\theta$
}
\end{figure}
\end{frame}




\subsection{Jets}

\begin{frame}{All jets}
\begin{figure}
\includegraphics[width=0.5\textwidth]{{\figurepath/comp_raw_jet_n}.pdf}
\includegraphics[width=0.5\textwidth]{{\figurepath/comp_raw_jet_etot}.pdf}\\ \tripleFigDistance
\includegraphics[width=0.5\textwidth]{{\figurepath/comp_pre_jet_n}.pdf}
\includegraphics[width=0.5\textwidth]{{\figurepath/comp_pre_jet_etot}.pdf}
\caption{
  Total jet number and energy
}
\end{figure}
\end{frame}

\begin{frame}{Jet kinematics}
\begin{figure}
\includegraphics[width=0.5\textwidth]{{\figurepath/comp_raw_jet_e}.pdf}
\includegraphics[width=0.5\textwidth]{{\figurepath/comp_raw_jet_pt}.pdf}\\ \tripleFigDistance
\includegraphics[width=0.5\textwidth]{{\figurepath/comp_pre_jet_e}.pdf}
\includegraphics[width=0.5\textwidth]{{\figurepath/comp_pre_jet_pt}.pdf}
\caption{
  Jet energy and \pt
}
\end{figure}
\end{frame}

\begin{frame}{Jet kinematics}
\begin{figure}
\includegraphics[width=0.5\textwidth]{{\figurepath/comp_raw_jet_pt_0}.pdf}
\includegraphics[width=0.5\textwidth]{{\figurepath/comp_raw_jet_pt_1}.pdf}\\ \tripleFigDistance
\includegraphics[width=0.5\textwidth]{{\figurepath/comp_pre_jet_pt_0}.pdf}
\includegraphics[width=0.5\textwidth]{{\figurepath/comp_pre_jet_pt_1}.pdf}
\caption{
  Leading and sub-leading jet \pt
}
\end{figure}
\end{frame}

\begin{frame}{Jet kinematics}
\begin{figure}
\includegraphics[width=0.5\textwidth]{{\figurepath/comp_raw_jet_theta}.pdf}
\includegraphics[width=0.5\textwidth]{{\figurepath/comp_raw_jet_phi}.pdf}\\ \tripleFigDistance
\includegraphics[width=0.5\textwidth]{{\figurepath/comp_pre_jet_theta}.pdf}
\includegraphics[width=0.5\textwidth]{{\figurepath/comp_pre_jet_phi}.pdf}
\caption{
  Jet $\phi$ and $\theta$
}
\end{figure}
\end{frame}





% \subsection{Derived observables}

% \begin{frame}{Invariant masses}
% \begin{figure}
% \includegraphics[width=0.5\textwidth]{{\figurepath/comp_raw_mjj}.pdf}
% \includegraphics[width=0.5\textwidth]{{\figurepath/comp_raw_mln}.pdf}\\ \tripleFigDistance
% \includegraphics[width=0.5\textwidth]{{\figurepath/comp_pre_mjj}.pdf}
% \includegraphics[width=0.5\textwidth]{{\figurepath/comp_pre_mln}.pdf}
% \caption{
%   Jet and charged lepton--neutrino invariant mass
% }
% \end{figure}
% \end{frame}

% \begin{frame}{Total event}
% \begin{figure}
% \includegraphics[width=0.5\textwidth]{{\figurepath/comp_raw_mtot}.pdf}
% \includegraphics[width=0.5\textwidth]{{\figurepath/comp_raw_etot}.pdf}\\ \tripleFigDistance
% \includegraphics[width=0.5\textwidth]{{\figurepath/comp_pre_mtot}.pdf}
% \includegraphics[width=0.5\textwidth]{{\figurepath/comp_pre_etot}.pdf}
% \caption{
%   Total invariant mass and energy of the event
% }
% \end{figure}
% \end{frame}


% \section{Results}



\section{Conclusion}

\begin{frame}{Conclusion}
The two detector models show significant differences in
\begin{itemize}
\item beam energy: new is softer (should be the same, since MC events are identical)
% \item missing energy:
\item the lepton type: new has more muons and less electrons
\item the lepton energy spectrum: new is softer
\item the lepton $\theta$: new is less forward
\item the jet energy spectrum: new is harder
\item the jet $\theta$: new is more forward
\end{itemize}

Thank you for your attention. Feedback is very welcome!

\end{frame}


\appendix

\section{Appendix}

\begin{frame}{Event properties}
\begin{figure}
\includegraphics[width=0.5\textwidth]{{\figurepath/comp_raw_sf}.pdf}\\ \tripleFigDistance
\includegraphics[width=0.5\textwidth]{{\figurepath/comp_pre_sf}.pdf}
\caption{
  The applied scale factors
}
\end{figure}
\end{frame}







\end{document}
